\ihead{Anleitung zur LaTeX-Vorlage}
\chapter{Anleitung zur LaTeX-Vorlage}

Diese LaTeX-Vorlage ist eine einsatzbereite Arbeitsgrundlage für wissenschaftliche Arbeiten. Sie liefert eine klare Grundstruktur, einheitliche Formatierung und direkt nutzbare Beispiele für typische Inhalte wie Tabellen, Abbildungen, Code-Listings und Mermaid-Diagramme.

\section{Wofür die LaTeX-Vorlage gedacht ist}
Die LaTeX-Vorlage richtet sich an Hausarbeiten, Projektberichte, Bachelor- und Masterarbeiten. Ziel ist, dass Sie nicht bei Null starten müssen, sondern sofort mit einer sauberen Dokumentstruktur arbeiten koennen.

\section{Projektstruktur}
Die Dateien sind so organisiert, dass Inhalt und Konfiguration getrennt bleiben:

\begin{itemize}
    \item \texttt{dokument.tex}: Hauptdatei mit Präambel, Metadaten und Einbindung aller Kapitel.
    \item \texttt{meta/}: Vorspann (Titelblatt, Kurzfassung, Abkürzungsverzeichnis, Erklärungen).
    \item \texttt{kapitel/}: Fachinhalt der Arbeit, kapitelweise getrennt.
    \item \texttt{anhang/}: Anhangsinhalte wie zusätzliche Tabellen, Abbildungen und Code.
    \item \texttt{bib/}: Literaturdatenbank für Zitate und Literaturverzeichnis.
    \item \texttt{bilder/}: Grafiken und weitere visuelle Assets.
\end{itemize}

\section{Schneller Einstieg}
\begin{enumerate}
    \item Öffnen Sie \texttt{dokument.tex} und tragen Sie Ihre Metadaten ein.
    \item Passen Sie die Inhalte in \texttt{meta/} an (z.B. Titelblatt und Abkürzungen).
    \item Schreiben Sie den Fachtext in \texttt{kapitel/}.
    \item Ergänzen Sie Abbildungen in \texttt{bilder/} und verweisen Sie im Text darauf.
    \item Kompilieren Sie das Dokument (typisch: \texttt{pdflatex} $\rightarrow$ \texttt{bibtex} $\rightarrow$ \texttt{pdflatex} $\rightarrow$ \texttt{pdflatex}).
\end{enumerate}

\section{Was bereits vorbereitet ist}
\begin{itemize}
    \item Titelblatt, Inhaltsverzeichnis und Verzeichnisse für Abbildungen, Tabellen und Codebeispiele.
    \item Abkürzungsverwaltung über das \texttt{acronym}-Paket.
    \item Literaturverwaltung über \texttt{biblatex} mit vorhandener Beispiel-\texttt{.bib}-Datei.
    \item Beispielinhalte in Kapitel 2, inklusive Nutzwertanalyse sowie Mermaid-Flowchart und Mermaid-Gantt.
\end{itemize}